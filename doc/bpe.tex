\section{BPE: Business Processes}

\subsection{Overview}
BPE is Synrc Cloud Stack Erlang Application that bring Erlang for for Enterprise.
It provides infrastructure for Workflow Definitions, Process Orchestration,
Rule Based Production Systems and Distributed Storage.

\subsection{Pi Calculus, Petri nets and Reactive systems}

\subsection{Typing Pi Calculus}

\subsection{Systems}

\subsection{Scenarios}
Workflows are complimentary to business rules and could be specified separetly.
BPE defenitions provides front API to the end-user application.
Workflow Engine -- is an Erlang/OTP application which handles process definitions,
process instances, and provides very clean API for Workplaces.

\paragraph{}
Before using Process Engine you need to define the set of business process
workflows of your enterprise. This could be done via Erlang terms or some DSL
that lately converted to Erlang terms. Internally BPE uses Eralng terms
workflow definition:

\vspace{1\baselineskip}
\begin{lstlisting}
bpe:start(
   #process{name="Order11",
       flows=[ #sequenceFlow{source="begin",target="end2"},
               #sequenceFlow{source="end2",target="end"}],
       tasks=[ #userTask{name="begin"},
               #userTask{name="end2"},
               #endEvent{name="end"}],
       task="begin",beginEvent="begin",endEvent="end"},[]).
\end{lstlisting}


\newpage
\vspace{1\baselineskip}
\begin{lstlisting}
deposit_app() ->

    #process { name = 'Create Deposit Account',

        flows = [
            #sequenceFlow{source='Init',      target='Payment'},
            #sequenceFlow{source='Payment',   target='Signatory'},
            #sequenceFlow{source='Payment',   target='Process'},
            #sequenceFlow{source='Process',   target='Final'},
            #sequenceFlow{source='Signatory', target='Process'},
            #sequenceFlow{source='Signatory', target='Final'}
        ],

        tasks = [
            #userTask    { name='Init',      module = deposit },
            #userTask    { name='Signatory', module = deposit},
            #serviceTask { name='Payment',   module = deposit},
            #serviceTask { name='Process',   module = deposit},
            #endEvent    { name='Final'}
        ],

        beginEvent = 'Init',
        endEvent = 'Final',
        events = [
             #messageEvent{name="PaymentReceived"}
        ]
    }.
\end{lstlisting}

\subsection{Actions}
Business rules should be specified by developers.
RETE is used for rules specifications which can be triggered during workflow.

\newpage

\subsection{BPMN 2.0}

The worklow definiton uses following persistent workflow model which is stored in KVS:


\vspace{1\baselineskip}
\begin{lstlisting}
-record(task,         { name, id, roles, module }).
-record(userTask,     { name, id, roles, module }).
-record(serviceTask,  { name, id, roles, module }).
-record(messageEvent, { name, id, payload }).
-record(beginEvent ,  { name, id }).
-record(endEvent,     { name, id }).
-record(sequenceFlow, { name, id, source, target }).
-record(history,      { ?ITERATOR(feed,true), name, task }).
-record(process,      { ?ITERATOR(feed,true), name,
                        roles=[], tasks=[], events=[], 
                        history=[], flows=[], rules,
                        docs=[], task, beginEvent, 
                        endEvent }).
\end{lstlisting}

Full set of BPMN 2.0 fields could be obtained at \footahref{http://www.omg.org/spec/BPMN/2.0}{OMG definition document, page 3-7}.

\newpage

\subsection{Sample Process}

\subsection{Erlang Session}

\vspace{1\baselineskip}
\begin{lstlisting}
(bpe@127.0.0.1)1> kvs:join().
ok

(bpe@127.0.0.1)1> rr(bpe).
[beginEvent,container,endEvent,history,id_seq,iterator,
 messageEvent,process,sequenceFlow,serviceTask,task,userTask]

(bpe@127.0.0.1)2> bpe:start(#process{name="Order11",
         flows=[#sequenceFlow{source="begin",target="end2"},
                #sequenceFlow{source="end2",target="end"}],
         tasks=[#userTask{name="begin"},
                #userTask{name="end2"},
                #endEvent{name="end"}],
         task="begin",beginEvent="begin",endEvent="end"},[]).
bpe_proc:Process 39 spawned <0.12399.0>
{ok,<0.12399.0>}

(bpe@127.0.0.1)3> gen_server:call(pid(0,12399,0),{complete}).

(bpe@127.0.0.1)4> gen_server:call(pid(0,12399,0),{complete}).

(bpe@127.0.0.1)5> gen_server:call(pid(0,12399,0),{complete}).

(bpe@127.0.0.1)5> bpe:history(39).
[#history{id = 28,version = undefined,container = feed,
          feed_id = {history,39},
          prev = 27,next = undefined,feeds = [],guard = true,
          etc = undefined,name = "Order11",
          task = {task,"end"}},
 #history{id = 27,version = undefined,container = feed,
          feed_id = {history,39},
          prev = 26,next = 28,feeds = [],guard = true,etc = undefined,
          name = "Order11",
          task = {task,"end2"}},
 #history{id = 26,version = undefined,container = feed,
          feed_id = {history,39},
          prev = undefined,next = 27,feeds = [],guard = true,
          etc = undefined,name = "Order11",
          task = {task,"begin"}}]
\end{lstlisting}


