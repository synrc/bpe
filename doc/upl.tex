\section{UPL: Processing Language}

\subsection{History}

Pocessing systems are using manually crafted application
relays to handle card processing business rules. Being defined by business
analysts these rules are fallen down to engineering teams informally.
Approach we provide pushes card processing to solid background in a form
of domain specific language common for all card plans analytics departments.

Having compact language we can formally build various translators
for particular customers and existing processing systems. At the same time
we provide reference back-end Erlang system implementation
for transactions processing. Also DSL gives us a natural and easy
verification strategies and compactifications.

\vspace{1\baselineskip}
\begin{lstlisting}[caption=Deposit Program]
    program Deposit_Plus UAH
    include 'PB-CASHBACK'
    deposit duration range monthly 1 -> 20%
                           monthly 3 -> 22%
                           monthly 6 -> 22%
                           annual 23%
            withdraw disabled
            auto
            charge enabled monthly limit max 20000
            monthly 1% of amount to account 'bonus'
            monthly 15% name 'tax' of deposit 
                        to account 'users/:client/tax'
\end{lstlisting}
\vspace{1\baselineskip}

This language could be easily extended to other domain areas like
internet payment processing, shopping mall bonus programs, mobile
operators tariff plans.

\newpage
\subsection{Objectives}

The aim is to create small and compact language for payment
transaction processing. Underlying instrumentation
code should be KVS layer for storing transaction chains but
naturally should be extended to different backends like Java,
PL/SQL and other languages currently involved in banking card processing.
We have several criteria to satisfy:

\paragraph{}
\begin{tabular}{ll}
     {\bf English} & Self-explanatory \\
     {\bf Fasten } & Time-to-market \\
     {\bf Optimized} & Minimal Back-end Operations \\
     {\bf Verified} & No regular bugs. Only business logic. \\
     {\bf Taxonomy} & Sane structure for extensions
\end{tabular}

\subsection{Operations}

\paragraph{}User Creation:

\begin{lstlisting}
    prepare user ':client' 
        name     ':fullname'
        age      ':birth'
        phone    ':ph'
        document ':passport'
        accounts 
            credit '/users/:client/credit' program 'PB-UNIVERSAL'
            account '/users/:client/:acc' program ':tariff'
\end{lstlisting}

\paragraph{}Process Transaction:

\begin{lstlisting}
    prepare transaction from account 'users/:client' 
                          to account ':beneficiar'
\end{lstlisting}

\paragraph{}Notifying:

\begin{lstlisting}
    prepare event 'users/:client'
\end{lstlisting}

\newpage
\subsection{Programs}

Programs are tariff programs, set of rules that we plug to transaction processing.
It feels like set of filters triggered each time we fire money
movements on account with a given card defenition.

\vspace{1\baselineskip}
\begin{lstlisting}[caption=BNF]

    Program = program Name Currency Forms

    Form = limit Amount
         | grace Amount days
         | credit CreditRules
         | rate ChargeRule
         | version Amount
         | deposit DepositRules
         | accounts AccountList
\end{lstlisting}
\vspace{1\baselineskip}

Example:

\vspace{1\baselineskip}
\begin{lstlisting}[caption=credit.card]
    program PLA_DEB USD
    limit 20000
    version 1.0
    credit monthly 10%
\end{lstlisting}
\vspace{1\baselineskip}


Programs are stored in its own space.

\vspace{1\baselineskip}
\begin{lstlisting}
    /programs/PB-UNIVERSAL.card
    /programs/PB-DEPOSIT-PLUS.card
    /programs/API.code
    /programs/UA.user
\end{lstlisting}
\vspace{1\baselineskip}

\newpage
\subsection{Language Forms}

Top level tariffs of billing rules are pluggable slangs that
share some common part of the languages. These common part we will
call language forms.

\vspace{1\baselineskip}
\begin{lstlisting}[caption=BNF]

       Direction = charge 
                 | withdraw

      ChargeRule = Fixed + Percent
                   of <amount | debt | credit | deposit | rate>
                   limit <min Amount> | <max Amount>
                   name Name
                   to account Name

    Periodically = monthly Amount 
                 | monthly Months -> ChargeRule
                 | daily ChargeRule
                 | annual ChargeRule

         Account = <credit | rate | deposit> Name

\end{lstlisting}
\vspace{1\baselineskip}

\subsection{Accounts}

Enterprise Tree handles clients, accounts, transactions, programs, events.
Programs could be assigned to each node and fires atomatically on access.

\vspace{1\baselineskip}
\begin{lstlisting}
    /personal/:client
    /personal/:client/bonus
    /personal/:client/credit
    /personal/:client/deposit
    /personal/:client/rate
\end{lstlisting}
\vspace{1\baselineskip}

\subsection{External Services}

External service has its own endpoints, and could be
addressed/mounted? to local system.

\vspace{1\baselineskip}
\begin{lstlisting}
    /external/visa/:client
    /external/master/:client
    /external/swift/:client
    /internet/paypal/:client
    /bonus/:client
\end{lstlisting}
\vspace{1\baselineskip}

\subsection{Transactions}

Transactions are stored per each client's account.

\vspace{1\baselineskip}
\begin{lstlisting}
    /personal/:client/transactions
\end{lstlisting}
\vspace{1\baselineskip}

\subsection{Deposit}

Deposit program forms ususally provides such attributes of
account as duration, rate, withdraw locking, charge
limits, fee options and other deposit specific options.
Deposit forms usually have "deposit" account name.


\vspace{1\baselineskip}
\begin{lstlisting}
    Enabled = enabled | disabled

    Deposit = duration Periodically
            | duration range [Periodically]
            | withdraw Enabled
            | charge Enabled | charge Enabled Periodically
            | auto
            | final Periodically move from Id to Id
            | fee ChargeRule
            | Periodically
\end{lstlisting}
\vspace{1\baselineskip}

\subsection{Credit}

Credit programs forms mainly provide transaction
filtering and other default account name "credit".

\vspace{1\baselineskip}
\begin{lstlisting}
    Credit = transaction [TransRule]
           | sratus Enabled Ammount
           | Periodically

 TransRule = cashin  Amount
           | wire    ChargeRule
           | cashout Amount

\end{lstlisting}
\vspace{1\baselineskip}
